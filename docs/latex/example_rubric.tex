\documentclass[letterpaper,12pt,addpoints]{exam}
\usepackage[utf8]{inputenc}
\usepackage[english]{babel}

\usepackage[top=0.5in, bottom=1in, left=0.75in, right=0.75in]{geometry}
\usepackage{amsmath,amssymb}
\usepackage{mdframed}
\usepackage{graphicx}

\pagestyle{headandfoot}
\firstpageheader{}{}{}
\firstpagefooter{}{\fontfamily{phv}\selectfont\thepage\ of \numpages}{\fontfamily{phv}\selectfont 8A23B1CC5}
\runningfooter{\fontfamily{phv}\selectfont Kontinua 26 (3/5)}{\fontfamily{phv}\selectfont\thepage\ of \numpages}{\fontfamily{phv}\selectfont 8A23B1CC5}


\begin{document}

{\fontfamily{phv}\selectfont

\textbf{Kontinua 26 (3/5)}

\Large \textbf{The Physics of Gases}

\vspace{2mm}

Scoring Rubric
} 
\parbox{0.35\textwidth}{

} %Ends helvetica


\begin{enumerate}

\item (3 points) In a particular game, a fair die is tossed.  If the number of spots showing is either 4 or 5 you win \$1, if the number of spots showing is 6 you win \$4, and if the number of spots showing is 1, 2, or 3 you win nothing.  Let X be the amount that you win. 
What is the expected value of X?

\vspace{0.2cm}
\begin{center}
\includegraphics[width=2cm]{dice_simple.png}
\end{center}

\begin{minipage}[t]{0.50\textwidth}
\textit{Answer:} You must enumerate the possibilities:
\begin{itemize}
\item There is a probability of 1/3 that you will win \$1.  
\item There is a probability of 1/6 that you will win \$4.
\item There is a probability of 1/2 that you will win \$0.
\end{itemize}

So:

$$\mathbb{E}[X] = \frac{1}{3}(1) + \frac{1}{6}(4) +  \frac{1}{2}(0) = \frac{6}{6} = 1 $$

\end{minipage}
\hspace{0.05\textwidth}
\begin{minipage}[t]{0.40\textwidth}

\textit{Rubric:} Give two points if the student set up the sum correctly.  

Give one additional point if the student did the arithmetic correctly. 
\end{minipage}

\item (4 points) The weight of written reports produced in a certain department has a Normal distribution with mean 60 g and standard deviation 12 g.  What probability that the next report will weigh less than 48 g?

\textit{Answer:}  50\% of the reports will weigh more than 60 g.  48g is one standard deviation from the mean, so 34.1\% of the reports will weigh between 48 and 60 grams.  Thus,  84.1\% of the reports weigh more than 48 grams.  Thus 15.9\% of the reports will weigh less than 48 grams.

There is a 15.9\% chance that the next report will weigh less than 48 grams.

\item (4 points) Complete these equations: (2 points per equation)

\begin{enumerate}
\item (2 points) 
\begin{equation*}
\frac{2}{5} + \frac{3}{7} = 
\end{equation*}

\begin{minipage}[t]{0.50\textwidth}
\textit{Answer:}  First you must find the common divisor:

$$\frac{2}{5} + \frac{3}{7} = \frac{14}{35} + \frac{15}{35} =   \frac{29}{35}$$

This cannot be reduced any further.

\end{minipage}
\hspace{0.05\textwidth}
\begin{minipage}[t]{0.40\textwidth}
\textit{Rubric:}  Give the student 1 point for finding the common denominator.  

Give the student an additional point for doing the arithmetic and recognizing that $\frac{29}{35}$ could not be reduced further.
\end{minipage}

\item (2 points)
\begin{equation*}
\frac{2}{5} + \frac{3}{7} = 
\end{equation*}

\begin{minipage}[t]{0.50\textwidth}

\textit{Answer:}  First you must find the common divisor:

$$\frac{2}{5} + \frac{3}{7} = \frac{14}{35} + \frac{15}{35} =   \frac{29}{35}$$

This cannot be reduced any further.

\end{minipage}
\hspace{0.05\textwidth}
\begin{minipage}[t]{0.40\textwidth}
\textit{Rubric:}  Give the student 1 point for finding the common denominator.  

Give the student an additional point for doing the arithmetic and recognizing that $\frac{29}{35}$ could not be reduced further.
\end{minipage}

\end{enumerate}

\item A reservation service receives requests according to a Poisson process with an mean rate of 6 per minute.
\begin{enumerate}
\item (3 points) What is the probability that during a given 1-min period the service center receives 3 requests?

\begin{minipage}[t]{0.50\textwidth}

\textit{Answer:} The Poisson distribution for $k$ events in a period, given the average rate of events is $\lambda$ is given by:

$$P(k) = \frac{\lambda^{k}e^{-\lambda}}{k!}$$

Substituting in:

$$P(3) = \frac{6^{3}e^{-6}}{3!} =  \frac{216e^{-6}}{6} = \frac{36}{e^6}$$

That is a perfectly good answer.  If you used a calculator, you could give an approximation:

$$P(3) = \frac{6^{3}e^{-6}}{3!} = \frac{36}{e^6} \approx 0.089$$

The answer, then, would be "There is an 8.9\% chance that the center would get exactly 3 calls in a minute.

\end{minipage}
\hspace{0.05\textwidth}
\begin{minipage}[t]{0.40\textwidth}
\textit{Rubric:} 

\begin{tabular}{c|p{1.5in}}
1 & Knowing the  PDF for the Poisson distribution \\
\hline
1 & Knowing what to substitute in where\\
\hline
1 & Doing the math and getting a correct answer
\end{tabular}
\end{minipage}

\item (3 points) What is the probability that during a given 1-min period, exactly four of the five operators receive no requests?(\textit{Hint}: treat either as a binomial process of 5 trials with 4 successes or consider 5 combinations of Poisson processes, e.g. only 1st operation receives a request  or only 2nd operation receives a request and so on)
\end{enumerate}

\end{enumerate}
\end{document}
